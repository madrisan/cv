%%%%%%%%%%%%%%%%%%%%%%%%%%%%%%%%%%%%%%%%%%%%%%%%%%%%%%%%%%%%%%%%%%%%%%%%%%%%%%%%%%%%
%%% "Curriculum vitae et studiorum" --- Davide MADRISAN
%%% (PlainTeX format)
%%%%%%%%%%%%%%%%%%%%%%%%%%%%%%%%%%%%%%%%%%%%%%%%%%%%%%%%%%%%%%%%%%%%%%%%%%%%%%%%%%%%

%%% Paging and style
\magnification\magstephalf
\hsize = 14.4truecm \vsize = 23.3truecm
\hoffset = 0.6truecm \voffset = 0.5truecm
\topskip = 0.8truecm
\parindent = 0.5truecm \parskip = .3\baselineskip
\tolerance = 400
\headline = {%
    \ifnum\pageno > 1
    \hbox to \hsize{%
        \hfil\nineit Davide Madrisan -- Curriculum vit\ae{} et studiorum\hfil}%
    \else\hbox to \hsize{\hfill}\fi}
%
\newfam\scfam                                                     % small cap family
\newfam\bsfam                                              % slanted boldface family
\newfam\tbfam                                               % typewriter bold family
\newfam\sffam                                                    % sans serif family
\newfam\sfbfam                                              % sans serif bold family
\newfam\sfifam                                            % sans serif italic family
\newskip\ttglue
%
\font\fiverm = cmr5
\font\fivei  = cmmi5
\font\fivesy = cmsy5
\skewchar\fivei = '177\skewchar\fivesy = '60
%
\font\sixrm = cmr6
\font\sixit = cmti6
\font\sixbf = phvb8rn at 6pt  % cmbx6
\font\sixbi = cmmib6
\font\sixi  = cmmi6
\font\sixsy = cmsy6
\skewchar\sixi = '177\skewchar\sixsy = '60
%
\font\eightit = cmti8
\font\eightrm = cmr8
%
\font\ninei  = cmmi9
\font\ninesy = cmsy9
\font\nineex = cmex9
\font\ninebi = cmmib9
\font\ninerm = cmr9
\font\nineit = cmti9
\font\ninesl = cmsl9
\font\ninett = cmtt9 % pcrr8r at 9pt [can't be generated by KDVI provided by QiLinux 1.0alpha2]
\font\ninetb = pcrb8r at 9pt
\font\ninebf = cmbx9
\font\ninesc = cmcsc9
\font\ninebs = cmbxsl9
\font\niness = phvr8r at 9pt  % cmss9
\font\ninesb = phvb8r at 9pt  % cmssbx9
\font\ninesi = cmssi9
\skewchar\ninei = '177\skewchar\ninesy = '60
\hyphenchar\ninett = -1
\def\ninepoint{%
    \def\rm{\fam0\ninerm}%
        \textfont0 = \ninerm\scriptfont0 = \sixrm\scriptscriptfont0 = \fiverm%
    \textfont1 = \ninei \scriptfont1 = \sixi  \scriptscriptfont1 = \fivei%
    \textfont2 = \ninesy\scriptfont2 = \sixsy \scriptscriptfont2 = \fivesy%
    \textfont3 = \nineex\scriptfont3 = \nineex\scriptscriptfont3 = \nineex%
    \def\it{\fam\itfam\nineit}\textfont\itfam = \nineit%
    \def\sl{\fam\slfam\ninesl}\textfont\slfam = \ninesl%
    \def\tt{\fam\ttfam\ninett}\textfont\ttfam = \ninett%
    \def\tb{\fam\tbfam\ninetb}\textfont\tbfam = \ninetb%
    \def\bf{\fam\bffam\ninebf}\textfont\bffam = \ninebf%
    \def\sc{\fam\scfam\ninesc}\textfont\scfam = \ninesc%
    \def\bs{\fam\bsfam\ninebs}\textfont\bsfam = \ninebs%
    \def\sf{\fam\sffam\niness}\textfont\sffam = \niness%
    \def\sfb{\fam\sfbfam\ninesb}\textfont\sfbfam = \ninesb%
    \def\sfi{\fam\sfifam\ninesi}\textfont\sfifam = \ninesi%
    \tt\ttglue = .5em plus.25em minus.15em%
    \normalbaselineskip = 11pt%
    \setbox\strutbox = \hbox{\vrule height8pt depth3pt width0pt}%
    \normalbaselines\rm}
\ninepoint % default font size
%\ithyph
%%%%%%%%%%%%%%%%%%%%%%%%%%%%%%%%%%%%%%%%%%%%%%%%%%%%%%%%%%%%%%%%%%%%%%%%%%%%%%%%%%%%

\def\personaldata#1{\hbox to\hsize{%
    \baselineskip = 14truept%
    \vtop{\halign{{\rm##}\hfil&\qquad\qquad\strut##\hfil\cr#1\crcr}}\hfill}}
\def\personaldataline#1#2{%
  %\line{\hbox to 45mm{{\it #1\/}:\hss}\hbox{#2}}\cr}
   \line{\hbox to 45mm{{\it #1\/}\hss}\hbox{#2}}\cr}

\def\endpoint#1{\vrule height #1pt width #1pt depth 0pt}
\def\sendpoint{\endpoint{7}}
\def\section#1{%
    \penalty-600\vskip 10truemm plus2truemm minus2truemm%
    \leftline{\sendpoint\quad\sectionfont#1}%
    \nobreak\vskip 3truemm plus 1truemm minus 1truemm}
\def\subsection#1{%
    \penalty-600\vskip 5truemm plus1truemm minus1truemm%
    \leftline{\subsectionfont#1}%
    \nobreak\vskip 1truemm plus 1truemm minus 1truemm}
\font\smallsy = cmsy6
\def\bitem#1{%
    \item{\raise1.2pt\hbox{$\textfont2 = \smallsy\bullet$}}{%
       {\it #1\/}:~}\nobreak\ignorespaces}
\def\bitemunamed{%
    \item{\raise1.2pt\hbox{$\textfont2 = \smallsy\bullet$}}\ignorespaces}
\def\bitemitemunamed{%
    \itemitem{\raise1.2pt\hbox{$\textfont2 = \smallsy\bullet$}}\ignorespaces}
%\def\bdot{$\,\cdot\,$}
\def\bdot{\raise1.2pt\hbox{$\textfont2 = \smallsy\bullet$}}
\font\smallr = cmr6
\font\smallsl = cmsl6
\def\plusplus{\nobreak\raise1pt\hbox{$\textfont0 = \smallr+\kern-.06667em+$}}

%\def\smallplus{\nobreak\raise.16em\hbox{$\textfont0 = \smallr+$}}
\def\smallplus{\nobreak\raise.16em\hbox{\fiverm+}}
%
\font\tenmsa = msam10
\font\sevenmsa = msam7
\font\fivemsa  = msam5
\newfam\msafam
\def\undefine#1{\let#1\undefined}
\catcode`! = 11
\def\!hexnumber#1{%
    \ifcase#1 0\or 1\or 2\or 3\or 4\or 5\or 6\or 7\or 8\or9
        \or A\or B\or C\or D\or E\or F\fi}
\def\newsymbol#1#2#3#4{%
    \mathchardef#1 = "#2\!msafam#3#4}
\edef\!msafam{\!hexnumber\msafam}    % numero della famiglia dei font in esadecimale
\mathchardef\leq = "3\!msafam36
\def\ams{\fam\msafam = \tenmsa}%
    \textfont\msafam = \tenmsa%
    \scriptfont\msafam = \sevenmsa \scriptscriptfont\msafam = \fivemsa%
\catcode`! = 12
%
\font\titlefont = cmss12 % cmbx12 at14pt
\font\sectionfont = cmssbx10 % cmbx10
\font\subsectionfont = cmss9

%%%%%%%%%%%%%%%%%%%%%%%%%%%%%%%%%%%%%%%%%%%%%%%%%%%%%%%%%%%%%%%%%%%%%%%%%%%%%%%%%%%%
\def\today{%
   \the\day\space\ifcase\month%
      \or Janvier\or F\'evrier\or Mars\or Avril\or Mai\or Juin%
      \or Juillet\or Ao\^ut\or Septembre\or Octobre\or Novembre\or D\'ecembre\fi%
   \space\number\year}

%\def\dateitem#1{%
%   \par\noindent
%   \hbox to 78pt{\bf#1\hss\hskip 7pt}\hangindent 78pt\ignorespaces}
%\def\yskip {\vskip 1pt plus 1pt minus 0pt}
%\parindent = 8mm
\def\dateitem#1#2{%
   \par\hang\noindent\line{\bf#1\hss#2}\vskip 0pt\noindent\ignorespaces}

%\def\yskip {\vskip 1pt plus 1pt minus 0pt}

%\newbox\osqrbracket \newbox\csqrbracket
%\setbox\osqrbracket =
%    \hbox{\hskip.5pt\lower2pt\hbox{\vbox{\hrule height.3pt%
%    \hbox{\vrule height9pt width.3pt\hskip2.5pt}\hrule height.3pt}}}
%\setbox\csqrbracket =
%    \hbox{\lower2pt\hbox{\vbox{\hrule height.3pt%
%    \hbox{\hskip2.5pt\vrule height9pt width.3pt}\hrule height.3pt}}\hskip.5pt}
%\def\labeltext#1{%
%    \nobreak\noindent\copy\osqrbracket{\it#1\/}\copy\csqrbracket%
%    \vskip -\parskip\noindent\ignorespaces}
%\def\labeltext#1{%
%    \nobreak\noindent{\bf#1\/}\vskip -\parskip\noindent\ignorespaces}

\def\certification#1#2#3{%
   \par\hang\noindent\line{\it#1 -- #2\hss}%
   \vskip -3pt\noindent\ignorespaces\hskip\parindent{\bf#3}}

%%%%%%%%%%%%%%%%%%%%%%%%%%%%%%%%%%%%%%%%%%%%%%%%%%%%%%%%%%%%%%%%%%%%%%%%%%%%%%%%%%%%

\vglue 1.0truemm
\line{\titlefont Curriculum Vit\ae{} et Studiorum\hfill\eightit\today}

\vskip 10truemm plus2truemm minus2truemm
%\rightline{\bf Expert en syst\`emes informatiques Open Source Linux et r\'eseaux}
\rightline{\bf%
Experienced IT Professional in Linux and OpenSource Development,}
\rightline{\bf%
Administration, Industrialization, Monitoring.}
\rightline{\bf%
Data Science, Big~Data, and Machine Learning passionate.}
\vskip 6truemm plus2truemm minus2truemm

%%%%%%%%%%%%%%%%%%%%%%%%%%%%%%%%%%%%%%%%%%%%%%%%%%%%%%%%%%%%%%%%%%%%%%%%%%%%%%%%%%%%

\personaldata{%
   \personaldataline{Name{\rm~:}}{Davide Madrisan}
   \personaldataline{Phone number{\rm~:}}{\smallplus 33.(0)4.93579201}
   \personaldataline{Email address{\rm~:}}{\rm davide\kern-.0pt{.}\kern-.0pt{madrisan}%
            \lower.3ex\hbox{@}gmail\kern-.0pt{.}\kern-.0pt{com}}
   \personaldataline{Nationality{\rm~:}}{Italian}
   \personaldataline{Usual Residence{\rm~:}}{Nice, France}
  %\personaldataline{Personal website{\rm~:}}{\rm https:/\negthinspace/sites.google.com/site/davidemadrisan}
   \personaldataline{Personal website{\rm~:}}{\tt https://sites.google.com/site/davidemadrisan}
   \personaldataline{                       }{\tt https://madrisan.wordpress.com}
   \personaldataline{GitHub Public Repositories{\rm~:}}{\tt https://github.com/madrisan}
   \personaldataline{Linkedin{\rm~:}}{\tt https://www.linkedin.com/in/madrisan}
}

\vskip 6truemm plus2truemm minus2truemm
\centerline{$\diamond\quad\diamond\quad\diamond$}
\vskip 3truemm plus2truemm minus2truemm

{\it\noindent
Starting with degrees in Applied Mathematics and a Bachelor in Music 
Performance and Composition and then moving on into Cisco WAN Networking and 
Network Security,
Linux System Development and Administration, Nagios/Centreon Monitoring and Linux 
Embedded System Development, 
I have a wealth of background knowledge, maturity, and ability to learn to bring to 
any new challenge or opportunity.

I'm now actively improving my skills in Statistics, Machine Learning,
(Big) Data Analysis and Technologies, Neural Networks, Genetic Algorithms,
GPGPU Programming (CUDA and OpenCL), Cloud Protocols, and Web Programming.}
%%%%%%%%%%%%%%%%%%%%%%%%%%%%%%%%%%%%%%%%%%%%%%%%%%%%%%%%%%%%%%%%%%%%%%%%%%%%%%%%%%%%

\section{EXPERIENCES}

\dateitem{July~2014 -- current}{Senior DevOps and Linux Engineer at Sopra-Steria}
I joined Sopra-Steria group as a L3 consulting Linux administrator and DevOps on
several Red~Hat based new projects.

\vskip -2mm
\subsection{Summary of my tasks}

\item{\bdot} Validate, improve and build High Availability solutions,
   for hosting web applications and Oracle databases, by using the RedHat clustering 
   technologies (Red Hat CMAN clusters);
\item{\bdot} Develop a framework for ensuring the quality of the build process
   and report a list of deviations from the company policies and the customer requests;
\item{\bdot} Improve the Xymon monitoring;
\item{\bdot}  Configure network services (Postfix, BIND Caching DNS, Squid);
\item{\bdot}  Integrate the customer's software applications on their production systems.

\vskip -2mm
\subsection{Technologies}

\item{\bdot} Linux RHEL/CentOS version 5, 6 and 7;
\item{\bdot} RedHat Cluster Suite;
\item{\bdot} Postfix mail server, BIND DNS server, Proxy Squid, NFS, SAN, awstats.

\subsection{Certifications}

\certification{The Johns Hopkins University}
   {Coursera}{Data Science Specialization}
\bitemitemunamed{The Data Scientist’s Toolbox;}
\bitemitemunamed{R Programming;}
\bitemitemunamed{Getting and Cleaning Data;}
\bitemitemunamed{Exploratory Data Analysis;}
\bitemitemunamed{Reproducible Research.}

%\medskip

\certification{Berkeley University \& Databricks}
   {edX}{Big Data XSeries}
\bitemitemunamed{Introduction to Big Data with Apache Spark;}
\bitemitemunamed{Scalable Machine Learning.}

\certification{Universit\'e de Strasbourg}
   {FUN}{Stochastic Evolutionary Optimisation.}
\certification{Stanford University}
   {Coursera}{Machine Learning.}

%\medskip
   
\certification{W3C}
   {edX}{Learn HTML5 from W3C.}
\certification{Universidad Carlos III de Madrid}
   {edX}{Introduction to Programming with Java.}
\certification{The Hong Kong University of Science and Technology}
   {edX}{Introduction to Java Programming.}
\certification{\'Ecole Polytechnique F\'ed\'erale de Lausanne}
   {Coursera}{Introduction \`a la programmation orient\'ee objet (en C\plusplus).}
\certification{University of Illinois at Urbana-Champaign}
   {Coursera}{Heterogeneous Parallel Programming.}

%\medskip

\certification{University of California, san Diego}
   {Coursera}{How to Learn:~powerful mental tools to help you master tough subjects.}
\certification{Ludwig-Maximilians Universit\"at M\"unchen}
   {Coursera}{Competitive Strategy.}

%%%%%%%%%%%%%%%%%%%%%%%%%%%%%%%%%%%%%%%%%%%%%%%%%%%%%%%%%%%%%%%%%%%%%%%%%%%%%%%

\bigskip
\dateitem{December 2014 -- January 2015}{Networking Architect}

\noindent
Design and setup of a secure local network for a new advertising agency in Nice.

\subsection{Technologies}

\item{\bdot} Cisco Adaptive Security Appliance (ASA) Firewall;
\item{\bdot} DELL LAN Switches
\item{\bdot} Synology NAS
\item{\bdot} Ethernet, Wi-Fi, VLANs, VoIP, VPN

%%%%%%%%%%%%%%%%%%%%%%%%%%%%%%%%%%%%%%%%%%%%%%%%%%%%%%%%%%%%%%%%%%%%%%%%%%%%%%%

\bigskip
\dateitem{October 2011 -- June 2014}
{Monitoring Architect and Team Leader at IBM}

\noindent
L3 Nagios and Centreon administrator and developer.
Technical Leader of the Nagios monitoring team (France and Poland).
Focal point of the monitoring technologies based on open source solutions and
software.

This work at IBM France has given me the opportunity to setup new solutions 
for monitoring operating systems, databases, applications, and network 
appliances by using a wide range of opensource technologies.

\subsection{Summary of my tasks as a L3 Nagios and Centreon team leader}

\item{\bdot} New project management (architecture, transition, planning, 
   realization, and support);
\item{\bdot} Design, implement, validate, and document new monitoring solutions,
   by cooperating with architects, managers, system, database, and application 
   administrators;
\item{\bdot} Develop shell scripts and C programs (mainly Linux system level
   programs);
\item{\bdot} Create rpm packages and administer the yum repositories containing
   the monitoring and system software (250+~source rpm packages and  
   2000~packages created from scratch);
\item{\bdot} Improve the security of the core and Nagios servers, apache 
   services and applications;
\item{\bdot} Administer/update the existing monitoring configuration 
   (5000+~hosts; 60000+~monitored services).

\subsection{Technologies}

\item{\bdot} Operating Systems: AIX, HP-UX, Linux, Solaris, VMware ESX/ESXi, Windows;
\item{\bdot} Databases: DB2, MySQL, MS SQL, Oracle, PostgreSQL;
\item{\bdot} Network and IT security appliances: Linux, Cisco, BalaBit Log Management, 
   QRadar;
\item{\bdot} Development languages: bash, C, Perl, Python, and PHP.

\subsection{Open Source development}

\noindent
Open source extensions to the 
Nagios Plugins official software for monitoring Linux servers and has ported
some of these plugins to other Unixes (AIX, FreeBSD, OpenBSD, Solaris, 
SunOS openindiana).
Source code available at: {\tt https://github.com/madrisan}

%%%%%%%%%%%%%%%%%%%%%%%%%%%%%%%%%%%%%%%%%%%%%%%%%%%%%%%%%%%%%%%%%%%%%%%%%%%%%%%

\bigskip
\dateitem{April 2013 -– September 2013}
   {System and multimedia development on the RPi}

\noindent
Porting of the Linux distribution openmamba 
({\rm http:/\negthinspace/www.openmamba.org}) to the Raspberry~Pi.
Integration of the last versions of the opensource projects XBMC 
(opensource Media Center software) and liquidsoap 
(a swiss-army knife for multimedia streaming).

Development of a multimedia software in Python (by using the XBMC Python API)
for extending the XBMC functionalities and playing remote video files based on
a remotely configurable playlist with optional timing commands.

Creation of a web interface (fork of Raspcontrol), for checking the RPi status
and configuring the WiFi interface, running on top of the lighttpd web server.

\subsection{Technologies}

\item{\bdot} Raspberry Pi hardware (model B);
\item{\bdot} XBMC 12.3 ``{\it Frodo\/}'';
\item{\bdot} The OCaml compiler and programming environment;
\item{\bdot} liquidsoap 1.1.1 and extensions (ocaml-voaacenc, ocaml-shine above all);
\item{\bdot} The lighttpd web server;
\item{\bdot} Development languages: Bash, Python, HTML, PHP, javascript.

%%%%%%%%%%%%%%%%%%%%%%%%%%%%%%%%%%%%%%%%%%%%%%%%%%%%%%%%%%%%%%%%%%%%%%%%%%%%%%%

\bigskip
\dateitem{November 2008 -– September 2011}{Linux System Administrator at IBM}

\noindent
Senior Linux system adminitrator and Nagios/Centreon Monitoring expert.

During my work at IBM SSO in France, I was assigned on multiple client Linux support.
My experiences at IBM helped me to add management, compliance and process skills 
to my profile. 

\subsection{Summary of my tasks as a L3 Linux System Administrator}

\item{\bdot} Changes and incidents management using ITIL tools and methods;
\item{\bdot} Linux level 3 support to client and internal teams;
\item{\bdot} ITIL and GDF environment;
\item{\bdot} On-call duty (nights and weekends);
\item{\bdot} Development of yum repositories containing all the sofware 
   written by the team and myself.

\smallskip\noindent
Volunteer member of the IBM Center of Excellence Linux, working on solutions of 
standardization and validation of new technologies for Linux inside IBM France and
for improving the quality of the delivery.

\subsection{Technologies}

\item{\bdot} RHEL (RedHat Linux Enterprise) 2--6 -- physical and virtual 
   servers (VMWare, HyperV);
\item{\bdot} RedHat and Linux HA (heartbeat and DRBD) clusters;
\item{\bdot} SLES (SuSE Enterprise Linux) 10 and 11;
\item{\bdot} IBM servers and IBM blades; Boot On San;
\item{\bdot} VMware ESX management;
\item{\bdot} Yum and RPM packaging;
\item{\bdot} Bash, Python, PHP, and C development languages.

%%%%%%%%%%%%%%%%%%%%%%%%%%%%%%%%%%%%%%%%%%%%%%%%%%%%%%%%%%%%%%%%%%%%%%%%%%%%%%%

\bigskip
\dateitem{July 2008 -– Octobre 2008}
   {C/C\plusplus development on Linux mobile at OpenPlug}

\noindent
Porting of several MSDN library functions and win32 code, libraries and tools
developed under Microsoft Visual C/C\plusplus, to the Moblin environment,
an open source Linux-based operating system and application stack for
Mobile Internet Devices, netbooks, nettops and embedded devices.

\subsection{Technologies}

\item{\bdot} Ubuntu Linux;
\item{\bdot} GNU gcc compiler, linker, and GNU autotools; SVN repositories;
\item{\bdot} Development languages: Shell scripting, C and C\plusplus.

\bigskip
\dateitem{January 2007 -– May 2008}
   {Linux and VMware Certified Engineer at IBM Turin}

\noindent
Senior Linux system adminitrator and VMware certified engineer.

\subsection{Summary of my tasks}

\item{\bdot} Definition of a procedure for the disaster recovery 
   (rebuild of remote servers from scratch, by using a Tivoli backup and a 
    clonezilla iso image booted via an RSA interface);
\item{\bdot} Move hundreds of of physical servers and virtualized server 
   images from the Turin datacenter to IBM Milan ones;
\item{\bdot} Install, configure and manage a hundred production servers;
\item{\bdot} Install vendor patches and implement the IBM security policies;
\item{\bdot} Fix the problems of ``local triangulation'' between frontend 
   and backend servers when using a Radware's load balancing solution;
\item{\bdot} Install, configure, and manage several VMware ESX servers 
   (version 3.0.x);
\item{\bdot} Development of some shell scripts for creating the servers 
   utilization reports (application log analysis).

\subsection{Technologies}

\item{\bdot} RHEL (RedHat Enterprise Linux) AS/ES 3/4, SuSE SLES 10; Clonezilla;
\item{\bdot} IBM TSM;
\item{\bdot} VMware Environment.
 
%%%%%%%%%%%%%%%%%%%%%%%%%%%%%%%%%%%%%%%%%%%%%%%%%%%%%%%%%%%%%%%%%%%%%%%%%%%%%%%

\bigskip
\dateitem{September 2003 -– December 2005}
   {Linux System Development at QiNet S.r.l.}

\noindent
Developer of the `{\it QiLinux\/}' GNU/Linux distribution for {\it i586} and
{\it ppc} architectures, the first Italian distribution
{\it created from scratch}.

\noindent
April 2006 -- December 2006 (9 months). 
Leader of the team that has developed QiLinux~2.0, the derived commercial 
products `{\it Tuga Desktop\/}', and the educational live~CD 
`{\it QiLinux Docet\/}' and `{\it Qiko Junior\/}'.

Between 2003 and 2006 I had an experience as a Linux distribution developer 
which involved me into many different activities ranging from the conception
and creation from sources of all the pieces and technologies of a modern Linux
distribution to the development and debugging in the Linux environment and 
the compliance with the available standards.

\subsection{Summary of my tasks}

\item{\bdot} Creation and updating of all the specfiles shipped by the QiLinux project;
\item{\bdot} Build of the software for all the supported architectures;
\item{\bdot} Integration in QiLinux of all the available Linux desktop and server technologies;
\item{\bdot} Products quality and usabiity analysis, testing and bug fixing;
\item{\bdot} Make the distribution conform to the FHS~2.3 and LSB~3.0 Linux standards;
\item{\bdot} Patching of the security issues and public release of the 
   `{\it QiLinux Security Advisories\/}';
\item{\bdot} Development of the autospec and libspec softwares for speed up the packages updates;
\item{\bdot} Development of the C software what was responsible for the creation
   of the packages dependency chart and the PHP pages in the QiLinux web portal;
\item{\bdot} Creation of the final ISO images of the opensource and commercial products;
\item{\bdot} Installation and support of QiLinux boxes.

\subsection{Technologies}

\noindent
All the Linux technologies ranging from the Linux kernel plus user-space 
related softwares (devfs, udev, hal, $\dots$) to the system tools, the local,
printing, and network services up to the development (autotools, GNU/gcc suite, 
LLVM, mono, OpenJDK, $\dots$) and desktop environments (KDE, Gnome, LXDE).

\nobreak
For more details,
\hfill\break\noindent
{\tt https:/\negthinspace/sites.google.com/site/davidemadrisan/linux}

%%%%%%%%%%%%%%%%%%%%%%%%%%%%%%%%%%%%%%%%%%%%%%%%%%%%%%%%%%%%%%%%%%%%%%%%%%%%%%%

\bigskip
\dateitem{March 2003 -– August 2003}
   {Cisco Network Specialist at Intesa Sanpaolo}

\noindent
Cisco LAN/WAN network administrator expert in one of the most important banks of Italy.

\subsection{Summary of my tasks}

\item{\bdot} Managing the local and WAN network infrastructure of the customer;
\item{\bdot} Proactively provide solutions to field resources to resolve the 
   customer's problem;
\item{\bdot} Install, configure, and managing an HP OpenView monitoring station;
\item{\bdot} Development of some scripts in the Windows environment for 
   processing alerts and sending emails.

\subsection{Technologies}

\item{\bdot} Cisco routers and switches;
\item{\bdot} HP OpenView monitoring.
 
%%%%%%%%%%%%%%%%%%%%%%%%%%%%%%%%%%%%%%%%%%%%%%%%%%%%%%%%%%%%%%%%%%%%%%%%%%%%%%%

\bigskip
\dateitem{October 2001 -– January 2003}
   {Cisco Network Expert Engineer at Atlanet}

\noindent
Cisco Engineer at Atlanet, FIAT's TLC venture (now part of the BT Telecom group), 
working in the provisioning team.
This experience helped me understand how to work in business critical environments 
and how to minimize the impact on the customer's production.

\subsection{Summary of my tasks}

\item{\bdot} Implement, troubleshoot, validate very large migration projects
   to the MPLS technology;
\item{\bdot} Configure and install Cisco routers on-site;
\item{\bdot} Troubleshoot the link connectivity issues with the telecom technicians;
\item{\bdot} Provide solutions to customer's network problems by using the Cisco
   technologies;
\item{\bdot} Create a software for generating the router configuration starting
   from a minimal set of specifications.

\subsection{Technologies}

\item{\bdot} Cisco routers with ADSL, ISDN, serial and Ethernet interfaces;
\item{\bdot} Hardware: Cisco routers 80X, 827, 160X-R, 1720, 1750, 261X, 
   262X, 3620, 3640, 366X
\item{\bdot} Software: Cisco IOS;
\item{\bdot} LAN protocols: DECnet, Ethernet, IBM SNA, TCP, Token Ring, UDP;
\item{\bdot} Routing protocols: EIGRP, OSPF, BGP;
\item{\bdot} Other network protocols/technologies: DLSw, ICMP, IP, L2TP, 
   MPLS, NAT, SNMP.

%%%%%%%%%%%%%%%%%%%%%%%%%%%%%%%%%%%%%%%%%%%%%%%%%%%%%%%%%%%%%%%%%%%%%%%%%%%%%%%

\bigskip
\dateitem{October 1999 -– September 2001}
   {Cisco Network Engineer at Fiat GlobalValue}

\noindent
Working either on-site or remotely for different Italian customers to develop,
troubleshoot and manage their Cisco WAN networks (Fiat ITS and GlobalValue, 
Sanpaolo, Miroglio Tessile).

\subsection{Summary of my tasks}

\item{\bdot} Troubleshoot and solve network problems (physical connectivity, 
   routing, LAN issues, $\dots$);
\item{\bdot} End-user support;
\item{\bdot} Develop small network related scripts for calculating IP parameters
   and doing SNMP queries on routers not accessible by the team.

\subsection{Technologies}

\item{\bdot} Cisco routers; Cisco IOS;
\item{\bdot} Network routing (EIGRP, OSPF, BGP) 
   and network protocols (telnet, ICMP, SNMP).
   
%%%%%%%%%%%%%%%%%%%%%%%%%%%%%%%%%%%%%%%%%%%%%%%%%%%%%%%%%%%%%%%%%%%%%%%%%%%%%%%

\bigskip
\dateitem{October 1998 -– February 1999}
   {\TeX\ development at Academy of Sciences}

\noindent
Development of a collection of TeX macros and document styles written in pure plainTeX
for the Royal cademy of Sciences of Turin.

The aim of this project was to develop a document style, using the plainTeX language, 
to easily write Mathematics and Physics papers by emulating the Microsoft Word 
Look\&Feel.
The most difficult part was implementing a complete support for the TrueType fonts,
ot natively available under plain\TeX.

See the result at:
\hfill\break\noindent
{\tt https:/\negthinspace/github.com/madrisan/TeX-macros/tree/master/astmacro}

%%%%%%%%%%%%%%%%%%%%%%%%%%%%%%%%%%%%%%%%%%%%%%%%%%%%%%%%%%%%%%%%%%%%%%%%%%%%%%%

\section{LANGUAGES}

\bitem {\it Italian}
Native language

\bitem {\it French}
Bilingual proficiency

\bitem {\it English}
Professional working proficiency

%%%%%%%%%%%%%%%%%%%%%%%%%%%%%%%%%%%%%%%%%%%%%%%%%%%%%%%%%%%%%%%%%%%%%%%%%%%%%%%

\section{EDUCATION}

\dateitem{2014--2015}{Coursera, edX and FUN MOOCs}
Lots of MOOCs on Data Science, parallel/WEB/generic programming and other domains.
\bigskip

\dateitem{Octobre 1994 -– July 1999}{Universit\`a degli Studi di Torino}
\noindent
Bachelor of Applied Science, Mathematics and Computer Science.
\bigskip

\dateitem{September 1982 -– June 1999}{Conservatorio di Musica G.~Verdi, Torino}
\noindent
Bachelor's degree, Music Performance, Piano, Composition.

%%%%%%%%%%%%%%%%%%%%%%%%%%%%%%%%%%%%%%%%%%%%%%%%%%%%%%%%%%%%%%%%%%%%%%%%%%%%%%%

\def\certification#1#2{%
   \par\hang\noindent\line{\rm#1 -- #2\hss}\vskip 0pt\noindent\ignorespaces}

\section{CERTIFICATIONS}

\certification{2007}{VMware Certified Professional on VI3 (VCP310).}
\certification{2003}{VPN-1/Firewall-1 Management II NG.}
\certification{2003}{ZyXEL Certified Network Engineer.}
\certification{2002}{CSPFA - Cisco Secure PIX Firewall Advanced.}
\certification{2001}{BSCN - Building Scalable Cisco Networks.}
\certification{2000 -- 2001}{CLSC Certified - Cisco LAN Switch Configuration.}

\vfill
\supereject\end
