%%% Curriculum vitae et studiorum
%%% Copyright (C) 2017-2019 by Davide Madrisan <davide.madrisan@gmail.com>

\input tex/cv.sty
\input tex/fontawesome.sty
\input tex/pdf-extentions.sty

%\input fontawesome.tex

\personaldata{%
   \personaldataline{\fafiveFlag}{Citizenship}{European Union -- Italy}
   \personaldataline{\fafiveHome}{Usual Residence}{Nice, France}
   % Master of Science in Mathematics
   % Doctor of Musical Arts (D.M.A.)
   \personaldataline{\fafiveTemple}{Education}{
            M.S. in Mathematics / DMA in Piano Performance}
   \personaldataline{\fafiveHandshake}{Languages}{
            Italian (native), French (fluent), English (good working knowledge)}
   \personaldataline{\fafiveLinkedIN}{
            LinkedIn Profile}{\url{linkedin.com/in/madrisan}}
   \personaldataline{\fafiveGitHub}{
            GitHub Public Repositories}{\url{github.com/madrisan}}
   \personaldataline{\fafiveWeb}{Personal Website}{
            \url{madrisan.com}}
   \personaldataline{\fafiveMail}{Email address}{\email}%
}

\vskip -10truemm

\line{\bigfont Davide Madrisan\hfill}
\medskip

\line{\titlefont{\color{darkgreen}Curriculum Vit\ae{} et Studiorum}%
%
\footnote{\rm$^{(\negthinspace*\negthinspace)}$}{%
    \eightit%
    The online version of this Curriculum Vit\ae{} is available at\/
    {\eighttt \url{github.com/madrisan/cv}}}%
%
{\quad\fiverm\today}\quad{\color{darkgreen}\hrulefill}}

\bigskip

\leftline{\bf\color{darkgray}%
Lead Infra DevOps and Automation Engineer}
\leftline{\bf\color{darkgray}%
Linux Open Source Developer with 20 years of experience}

{\narrower\medskip\noindent\it
Starting with a degree in Applied Mathematics and a Doctorate of Musical Arts
in Pia\-no, and then moving on into Cisco WAN networking and network security,
Linux di\-stri\-bu\-tion development,
Linux administrationi and Nagios/Centreon monitoring in highly demanding
production environments, Linux embedded systems, Web development,
and Linux DevOps technologies,
I~think I~have a wealth of background knowledge, maturity, and ability
to learn to bring to any new challenge or opportunity.
\medskip}

\subsection{Areas of Interest}

I'm currently busy playing with SaltStack, Docker, Go,
cloud technologies, and improving my knowledge in functional programming,
but my interests in technology and maths/physics fields cover the following
areas (somewhat ordered by decreasing degree of time occorded):

\item{\bdot}
    DevOps, Infrastructure \& Configuration as Code Technologies;
\item{\bdot}
    Maths, currently digging around Logic and Computation Theory,
    Complex and Dynamic Systems, Chaos Theory, Fractals;
%\item{\bdot}
%    Climate Global Changes threats (and their mathematical formalizations);
\item{\bdot}
    Functional, concurrent, and Web programming
    (in Python, Go, Scala, Node.js);
\item{\bdot}
    Cloud Computing Technologies, Microservices Architectures;
%\item{\bdot}
%    Physics: Cosmology, superstring theory and \hbox{M--theory};
\item{\bdot}
    Linux, Docker containers and Embedded Linux Systems.

\medskip

\noindent
An annotated scatter plot that should help visualizing my job and
learning activities.
% (I'm an MOOC-addict) activities}
% I'm a serial MOOC--taker (around $\sim$$60$ completed so far).

\medskip
\centerline{%
    \hbox{\pdfximage height 7cm{images/experiences.png}\pdfrefximage\pdflastximage}}

\vfill\eject

%%%%%%%%%%%%%%%%%%%%%%%%%%%%%%%%%%%%%%%%%%%%%%%%%%%%%%%%%%%%%%%%%%%%%%%%%%%%%%%%%%%%

\section{Work Experience}

\vskip-\bigskipamount\medskip
\dateitem{September~2017}{current}{Lead Infra DevOps~@~Qwant.com}

\stress{Infra Lead DevOps\/} at Qwant (\url{www.qwant.com}), the
French startup company that develops the eponymous web search engine that respects
the users' privacy.

\smallskip\noindent
\tasks
Software--define ({\it reasonably}\/)~{\it everything\/} with {\it SaltStack\/}:
from infrastructure to software integration, deployment, and testing.
Develop some Salt state/execution/SDB modules, and runners for a better
integration with GitLab~CI, HashiCorp~Vault HA (with HashiCorp~Consul and
Apache ZooKeeper as secure storage backends), \phpIPAM{}, and for deploying
virtual servers and Swarm clusters.
Code the deployment of web services, databases, Debian apt repositories, and the
software projects developed by the Qwant developers.

\smallskip\noindent
\software
\url{github.com/madrisan/debian-packages} ---
Debian Stretch Packages for Vault, Consul, py-zabbix, pyvmomi.

\noindent
\url{github.com/madrisan/hashicorp-vault-monitor} ---
HashiCorp Vault Monitoring Tool.

\noindent
\url{github.com/madrisan/keepalived-vault-ha} ---
Keepalived Tracking Script for an high availability HashiCorp Vault cluster.

\noindent
\url{github.com/madrisan/saltstack-mattermost} ---
A SaltStack extension module for interacting with Mattermost Incoming Webhooks.

\noindent
\url{github.com/madrisan/saltstack-phpipam} ---
A SaltStack extension module for interacting with a \phpIPAM{} server.

%\noindent
%\url{github.com/madrisan/saltstack-sdb-phpipam} ---
%SaltStack SDB module for interacting for a \phpIPAM{} server.

\smallskip\noindent
\technologies
Docker/Docker~Compose, Git/GitLab/GitLab~Runner, HAProxy,
HashiCorp Vault and Consul, Jenkins, MariaDB, NGiNX, \phpIPAM,
Phusion Passenger, SaltStack, Salt Cloud, Swarm, Unbound, uWSGI, ZooKeeper;
Golang, Python languages; CentOS and Debian Linux.

%%%%%%%%%%%%%%%%%%%%%%%%%%%%%%%%%%%%%%%%%%%%%%%%%%%%%%%%%%%%%%%%%%%%%%%%%%%%%%%

\dateitem{April~2013}{current}{\OS2--Open Source Solutions (micro entreprise)}

In parallel of my current job, I have decided to develop an additional professional
activity as Auto Entrepreneur, mainly focused on Open Source and Linux--based (new) technologies:

\item{\bdot} Web development (JavaScript/JQuery, MEAN Stack, Web Sites);
\item{\bdot} Linux Embedded system, multimedia, Python and web development on RaspberryPi;
\item{\bdot} LAN networking design and setup (Cisco, DELL, Synology NAS);
\item{\bdot} Web services based on AWS Cloud Technologies.

%%%%%%%%%%%%%%%%%%%%%%%%%%%%%%%%%%%%%%%%%%%%%%%%%%%%%%%%%%%%%%%%%%%%%%%%%%%%%%%

\dateitem{July~2014}{August~2017}{Senior DevOps and Linux Engineer at Sopra~Steria}

I joined Sopra~Steria group as a L3 consulting Linux administrator and DevOps on
several Red~Hat based new projects.

\smallskip\noindent
\tasks
Lead the switch to \stress{SaltStack\/} as primary tool for IT automation and system
configuration management, built from scratch.
Build, validate and improve \stress{High Availability\/} solutions,
for hosting web applications and Oracle databases, by using the
\stress{Red~Hat cluster suite\/} (cman, pacemaker/corosync).

\smallskip\noindent
\software
\url{github.com/madrisan/saltstack-code-snippets} ---
SaltStack code snippets for Linux.

\tinyskip\noindent
\url{github.com/madrisan/pyxymon} ---
improve the Xymon monitoring by developing new checks and fixing the
existing ones.

\tinyskip\noindent
Develop some scripts for building and packaging the infrastructure software
into dynamically created Docker containers, one per supported platform.
Develop a Python inventory tool for Linux using SaltStack as backend.

\tinyskip\noindent
\technologies
Linux~RHEL/CentOS/Debian, Red~Hat Cluster Suite, Docker containers,
AWS services, VMware. Git/GitHub/GitLab, Travis~CI. SaltStack and Spacewalk.
Shell and Python scripting languages, RWD (Bootstrap, CSS3, HTML5, AngularJS).

%%%%%%%%%%%%%%%%%%%%%%%%%%%%%%%%%%%%%%%%%%%%%%%%%%%%%%%%%%%%%%%%%%%%%%%%%%%%%%%

\dateitem{October 2011}{June 2014}
   {Monitoring Architect and Team Leader at IBM}

L3 Nagios and Centreon administrator and developer.
\stress{Technical Leader\/} of the \stress{Nagios monitoring team\/}
(France and Poland).
\stress{Focal point\/} of the monitoring technologies based on open source
solutions and software.
This work at IBM France has given me the opportunity to setup new solutions 
for monitoring operating systems, databases, applications, and network 
appliances by using a wide range of opensource technologies.

\smallskip\noindent
\tasks
New project management
(architecture, transition, planning, realization, and support).
Design, implement, validate, and document new monitoring solutions,
by cooperating with architects, managers, system, database, and application 
administrators.

\tinyskip
Administer all the monitoring configuration
(5k\smallplus~hosts; 60k\smallplus~monitored services), improve the security
of the core and Nagios servers, apache services and applications.

\tinyskip
Create rpm packages and administer the yum repositories containing
the monitoring and system software (250\smallplus~source rpm packages and
2000~packages created from scratch).

\tinyskip\noindent
\technologies
{\it Operating Systems\/}:
AIX, HP-UX, Linux, Solaris, VMware ESX/ESXi, Windows;
\par\noindent {\it Databases\/}: DB2, MySQL, MS SQL, Oracle, PostgreSQL;
\par\noindent {\it Network and IT security appliances\/}: Linux, Cisco,
BalaBit Log Management, QRadar;
\par\noindent {\it Languages\/}: bash, C, Perl, Python, HTPL, and PHP.

\smallskip\noindent
\software
\url{github.com/madrisan/nagios-plugins-linux} ---
extend the \stress{Nagios Plugins\/} official software for monitoring Linux
hosts and Docker containers.

Develop shell, Python scripts, and C (system level) programs.
Porting some of these C plugins to other Unixes 
(AIX, FreeBSD, OpenBSD, Solaris, SunOS openindiana).

%%%%%%%%%%%%%%%%%%%%%%%%%%%%%%%%%%%%%%%%%%%%%%%%%%%%%%%%%%%%%%%%%%%%%%%%%%%%%%%

%\bigskip
%\dateitem{April 2013}{September 2013}
%   {System and multimedia development on the RPi}
%Porting of the Linux distribution openmamba 
%(\url{www.openmamba.org}) to the Raspberry~Pi.
%Integration of the last versions of the opensource projects XBMC 
%(opensource Media Center software) and liquidsoap 
%(a swiss-army knife for multimedia streaming).
%
%Development of a multimedia software in Python (by using the XBMC Python API)
%for extending the XBMC functionalities and playing remote video files based on
%a remotely configurable playlist with optional timing commands.
%
%Creation of a web interface (fork of Raspcontrol), for checking the RPi status
%and configuring the WiFi interface, running on top of the lighttpd web server.
%
%\subsection{Technologies}
%
%\item{\bdot} Raspberry Pi hardware (model B);
%\item{\bdot} XBMC ``{\it Frodo\/}'' (now known as Kodi);
%\item{\bdot} The OCaml compiler, liquidsoap and extensions
%             (ocaml-voaacenc, ocaml-shine above all);
%\item{\bdot} The lighttpd web server;
%\item{\bdot} Development languages: Bash, Python, HTML, PHP, javascript.

%%%%%%%%%%%%%%%%%%%%%%%%%%%%%%%%%%%%%%%%%%%%%%%%%%%%%%%%%%%%%%%%%%%%%%%%%%%%%%%

\dateitem{November 2008}{September 2011}{L3 Senior Linux Sysadmin at IBM}

During my work at IBM SSO in France, I was assigned on multiple client Linux
support. My experiences at IBM helped me to add \stress{management},
\stress{compliance\/} changes and incidents management in a very strict
\stress{ITIL\/} and \stress{GDF\/} environment) and \stress{process skills\/}
to my profile. 

Volunteer member of the IBM \stress{Center of Excellence Linux}, working on
solutions of standardization and validation of new technologies for Linux
inside IBM France and for improving the quality of the delivery.

Development of yum repositories containing all the sofware written by the team
and myself, most of them for Nagios/Centreon monitoring purposes.

\tinyskip\noindent
\technologies
RHEL (Red~Hat) and SLES (SuSE) physical and virtual
servers (VMWare, HyperV),
Red~Hat and Linux HA (heartbeat and DRBD) clusters,
IBM servers and IBM blades, Boot On San,
VMware ESX management,
Yum and RPM packaging.
Bash, Python, PHP, and C development languages.

%%%%%%%%%%%%%%%%%%%%%%%%%%%%%%%%%%%%%%%%%%%%%%%%%%%%%%%%%%%%%%%%%%%%%%%%%%%%%%%

\dateitem{July 2008}{Septembre 2008}
   {C/C\plusplus{} development on Linux mobile at OpenPlug}

Porting of several MSDN library functions and win32 code, libraries and tools
developed under Microsoft Visual C/C\plusplus, to the Moblin environment,
an open source Linux-based operating system and application stack for
Mobile Internet Devices, netbooks, nettops and embedded devices.

%\subsection{Technologies}
%
%\item{\bdot} Ubuntu Linux;
%\item{\bdot} GNU gcc compiler, linker, and GNU autotools; SVN repositories;
%\item{\bdot} Development languages: Shell scripting, C and C\plusplus.

%%%%%%%%%%%%%%%%%%%%%%%%%%%%%%%%%%%%%%%%%%%%%%%%%%%%%%%%%%%%%%%%%%%%%%%%%%%%%%%

\dateitem{January 2007}{May 2008}
   {Linux and VMware Certified Engineer at IBM}

Senior Linux system adminitrator and VMware certified engineer at
IBM Turin.

%\subsection{Summary of my tasks}
%
%\item{\bdot} Definition of a procedure for the disaster recovery 
%   (rebuild of remote servers from scratch, by using a Tivoli backup and a 
%    clonezilla iso image booted via an RSA interface);
%\item{\bdot} Move hundreds of of physical servers and virtualized server 
%   images from the Turin datacenter to IBM Milan ones;
%\item{\bdot} Install, configure and manage a hundred production servers;
%\item{\bdot} Install vendor patches and implement the IBM security policies;
%\item{\bdot} Fix the problems of ``local triangulation'' between frontend 
%   and backend servers when using a Radware's load balancing solution;
%\item{\bdot} Install, configure, and manage several VMware ESX servers 
%   (version 3.0.x);
%\item{\bdot} Development of some shell scripts for creating the servers 
%   utilization reports (application log analysis).

%\subsection{Technologies}
%
%\item{\bdot} RHEL (Red~Hat Enterprise Linux) AS/ES 3/4, SuSE SLES 10; Clonezilla;
%\item{\bdot} IBM TSM;
%\item{\bdot} VMware Environment.

%%%%%%%%%%%%%%%%%%%%%%%%%%%%%%%%%%%%%%%%%%%%%%%%%%%%%%%%%%%%%%%%%%%%%%%%%%%%%%%
\vfill\eject

\dateitem{September 2003}{December 2006}
   {Linux System Development at QiNet S.r.l.}

Developer of the `\stress{QiLinux\/}' GNU/Linux distribution for
\stress{\it i586\/} and \stress{ppc\/} architectures, the first Italian
distribution \stress{created from scratch}.

April 2006 -- December 2006 (9 months). 
\stress{Leader of the team\/} that developed QiLinux~2.0, the derived commercial 
products `\stress{Tuga Desktop\/}', and the educational live~CD 
`\stress{QiLinux Docet\/}' and `\stress{Qiko Junior\/}'.

Between 2003 and 2006 I had an experience as a
\stress{Linux distribution developer\/}
which involved me into many different activities ranging from the conception
and creation from sources of all the pieces and technologies of a modern Linux
distribution to the development and debugging in the Linux environment and 
the compliance with the available standards.

\smallskip\noindent
\tasks
Creation and updates of all the \stress{specfiles\/} shipped by the QiLinux
project as a base for build the software for all the supported architectures.
\stress{Integration\/} in QiLinux of all the available Linux desktop and server
technologies.

Products \stress{quality\/}, \stress{usability\/} analysis,
\stress{bug fixing\/}, and conformance to the LSB standards.
Patching of the security issues and public release of the 
`\stress{QiLinux Security Advisories\/}'.
Creation of the final ISO images of the opensource and commercial products.

Installation and support of QiLinux boxes.

\noindent
\url{gitlab.mambasoft.it/openmamba/autospec} ---
development of the \stress{autospec\/} and \stress{libspec\/} software for
speed up the packages update, and make it possible the continuous
integration and testing of the new software in the mainstream distribution.

Development of the C software what was responsible for the creation
of the packages dependency chart and the PHP pages in the QiLinux web portal.

\tinyskip\noindent
\technologies
All the Linux technologies ranging from the Linux kernel plus user-space 
related software
% (devfs, udev, hal, D--Bus, $\dots$)
to the system tools, the local,
printing, and network services up to the development
% (autotools, GNU/gcc suite, LLVM, SVN, mono, OpenJDK, $\dots$)
and desktop environments. % (KDE, Gnome, LXDE).

%For more details:
%\url{sites.google.com/site/davidemadrisan/linux}.

%%%%%%%%%%%%%%%%%%%%%%%%%%%%%%%%%%%%%%%%%%%%%%%%%%%%%%%%%%%%%%%%%%%%%%%%%%%%%%%

\dateitem{March 2003}{August 2003}{Cisco Network Specialist at Intesa Sanpaolo}

Working as a Cisco LAN/WAN network administrator expert in one of the most
important banks of Italy.

%\subsection{Summary of my tasks}
%
%\item{\bdot} Managing the local and WAN network infrastructure of the customer;
%\item{\bdot} Proactively provide solutions for improving the customer's network;
%\item{\bdot} Install, configure, and administer an HP OpenView monitoring station;
%             develop some scripts in the Windows environment for processing alerts
%             and sending emails.

%\subsection{Technologies}
%
%\item{\bdot} Cisco routers and switches;
%\item{\bdot} HP OpenView monitoring.
 
%%%%%%%%%%%%%%%%%%%%%%%%%%%%%%%%%%%%%%%%%%%%%%%%%%%%%%%%%%%%%%%%%%%%%%%%%%%%%%%

\dateitem{October 2001}{January 2003}{Cisco Network Expert Engineer at Atlanet}

Cisco Engineer at Atlanet (FIAT's TLC venture now part of the BT Telecom group), 
working in a highly business and production critical environment.

Development of a fully featured software for generating Cisco router
configurations from a minimal set of specifications given by a network architect
to speed up and reduce the alea of the complex network migrations to the
MPLS network cloud topology.

%\subsection{Summary of my tasks}
%
%\item{\bdot} Implement, troubleshoot, validate very large migration projects
%   to the MPLS technology;
%\item{\bdot} Configure and install Cisco routers on-site and troubleshoot the
%   connectivity issues;
%\item{\bdot} Provide solutions to customer's network problems by using the Cisco
%   technologies;

%\subsection{Technologies}
%
%\item{\bdot} Cisco routers with ADSL, ISDN, serial and Ethernet interfaces;
%\item{\bdot} LAN protocols: DECnet, Ethernet, IBM SNA, TCP, Token Ring, UDP;
%\item{\bdot} Routing protocols: EIGRP, OSPF, BGP;
%\item{\bdot} Other network protocols/technologies: DLSw, ICMP, IP, L2TP, 
%   MPLS, NAT, SNMP.

%%%%%%%%%%%%%%%%%%%%%%%%%%%%%%%%%%%%%%%%%%%%%%%%%%%%%%%%%%%%%%%%%%%%%%%%%%%%%%%

\dateitem{October 1999}{September 2001}
   {Cisco Network Engineer at Fiat GlobalValue}

Working either on-site or remotely for different Italian customers (Fiat ITS
and GlobalValue, Sanpaolo, Miroglio Tessile) to develop, troubleshoot and
manage their Cisco WAN networks.

Develop small network related scripts for calculating IP parameters and doing
SNMP queries on routers not accessible by our team.

%\subsection{Technologies}
%
%Cisco routers, Cisco IOS, routing (EIGRP, OSPF, BGP),
%network (telnet, ICMP, SNMP) protocols.
   
%%%%%%%%%%%%%%%%%%%%%%%%%%%%%%%%%%%%%%%%%%%%%%%%%%%%%%%%%%%%%%%%%%%%%%%%%%%%%%%

\dateitem{October 1998}{February 1999}
   {\TeX{} development at Academy of Sciences}

Development of a collection of \TeX{} macros and document styles written in pure
\plainTeX{} for the \stress{Royal Academy of Sciences of Turin}.

Source code: \url{github.com/madrisan/TeX-macros/tree/master/astmacro}

The aim of this project was to develop a document style, using the \plainTeX{}
language, to easily write Mathematics and Physics papers by emulating the
Microsoft~Word Look\&Feel.
The most difficult part was implementing a complete support for the TrueType
fonts, not natively available under \plainTeX.

%%%%%%%%%%%%%%%%%%%%%%%%%%%%%%%%%%%%%%%%%%%%%%%%%%%%%%%%%%%%%%%%%%%%%%%%%%%%%%%

\section{Languages}

\bitem {\it Italian}
Native language

\bitem {\it French}
Bilingual proficiency

\bitem {\it English}
Professional working proficiency

%%%%%%%%%%%%%%%%%%%%%%%%%%%%%%%%%%%%%%%%%%%%%%%%%%%%%%%%%%%%%%%%%%%%%%%%%%%%%%%

\section{Education}

\vskip-\bigskipamount\medskip
\dateitem{September 2014}{}{Coursera, edX, Udacity, FUN, Complexity Explorer}

Lifelong learning for professional development and fun:
nearly 60 MOOCs completed so far, mainly on Data Science, Cloud Computing,
and parallel/generic/functional/web programming.

\vskip-\bigskipamount\medskip
\dateitem{October 1994}{July 1999}{Universit\`a degli Studi di Torino}

Master's degree in Applied Science, Mathematics and Computer Science.

\vskip-\bigskipamount\medskip
\dateitem{September 1982}{June 1993}{Conservatory of Music `G.~Verdi', Torino}

Doctor of Musical Arts in Piano.

%%%%%%%%%%%%%%%%%%%%%%%%%%%%%%%%%%%%%%%%%%%%%%%%%%%%%%%%%%%%%%%%%%%%%%%%%%%%%%%

\section{Courses and Certificates}

\certification{2016}
  {Full Stack Web Development Specialization -- {\it The Hong Kong University}.}
\certification{2016}
  {Paradigms of Computer Programming -- {\it Universit\'e catholique de Louvain}.}
\certification{2015}{Big Data XSeries Certificate -- {\it Berkeley University \& Databricks}.}
\certification{2007}{VMware Certified Professional on VI3 (VCP310).}
\certification{2003}{VPN-1/Firewall-1 Management II NG.}
\certification{2003}{ZyXEL Certified Network Engineer.}
\certification{2002}{CSPFA (Cisco Secure PIX Firewall Advanced).}
\certification{2001}{BSCN (Building Scalable Cisco Networks).}
\certification{2000}{CLSC (Cisco LAN Switch Configuration) Certified.}

\bigskip
\bgroup\eightpoint
\certgroup{BIG DATA and MACHINE LEARNING}

\certauthority{The Johns Hopkins University}{Coursera}
   \certspecialization{Data Science Specialization}
      \certitembullet{The Data Scientist’s Toolbox}
      \certitembullet{R Programming}
      \certitembullet{Getting and Cleaning Data}
      \certitembullet{Exploratory Data Analysis}
      \certitembullet{Reproducible Research}
\certauthority{Berkeley University \& Databricks}{edX}
   \certspecialization{Big Data XSeries}
      \certitembullet{Introduction to Big Data with Apache Spark}
      \certitembullet{Scalable Machine Learning}
\certitemoneline{Universit\'e de Strasbourg}{FUN}
   {Stochastic Evolutionary Optimization}
\certitemoneline{Stanford University}{Coursera}{Machine Learning}
\certitemoneline{Microsoft}{edX}{Introduction to R}
\certitemoneline{Hardvard University}{edX}
   {Statistics and R for Life Sciences}

\smallskip
\certgroup{BUSINESS}

\certitemoneline{Ludwig--Maximilians Universit\"at M\"unchen}{Coursera}
   {Competitive Strategy}

\smallskip
\certgroup{CLOUD COMPUTING and DEVOPS}

\certauthority{University of Illinois at Urbana-Champaign}{Coursera}
   \certitembullet{Cloud Computing Applications}
   \certitembullet{Cloud Networking}
\certitemoneline{AWS Training}{Amazon Web Services}
   {AWS Technical Essentials (version August 2016)}
\certauthority{Linux Foundation}{edX}
   \certitembullet{LFS132.x Introduction to Cloud Foundry and Cloud Native Software Architecture}
   \certitembullet{LFS151.x Introduction to Cloud Infrastructure Technologies}
   \certitembullet{LFS152.x Introduction to OpenStack}
   \certitembullet{LFS158.x Introduction to Kubernetes}
   \certitembullet{LFS161.x Introduction to DevOps: Transforming and Improving Operations}
\certitemoneline{Red~Hat}{edX}
   {DO081x Fundamentals of Containers, Kubernetes, and Red~Hat OpenShift}

\smallskip
\certgroup{MATHEMATICS}

\certauthority{Santa Fe Institue}{Complexity Explorer}
   \certitembullet{Introduction to Dynamical Systems and Chaos}
   \certitembullet{Fractals and Scaling}
   \certitembullet{Introduction to Renormalization}
   \certitembullet{Introduction to Computation Theory}

\smallskip
\certgroup{PROGRAMMING}

\certitemoneline{University of Illinois at Urbana-Champaign}{Coursera}
   {Heterogeneous Parallel Programming}
\certauthority{\'Ecole Polytechnique F\'ed\'erale de Lausanne}{Coursera}
   \certitembullet{Introduction \`a la programmation orient\'ee objet
             en C\plusplus}
   \certitembullet{Functional Programming Principles in Scala}
\certitemoneline{Universidad Carlos III de Madrid}{edX}
   {Introduction to Programming with Java (part~1)}
\certauthority{The Hong Kong University of Science and Technology}{edX}
   \certitembullet{Introduction to Java Programming (part~2)}
\certauthority{Universit\'e catholique de Louvain}{edX}
  \certitembullet{Paradigms of Computer Programming -- Fundamentals}
  \certitembullet{Abstraction and Concurrency}
\certitemoneline{University of Michigan}{Coursera}
   {Using Python to Access Web Data}

\smallskip
\certgroup{WEB PROGRAMMING}

\certitemoneline{W3C}{edX}{Learn HTML5 from W3C}
\certauthority{University of London}{Coursera}
   \certitembullet{Responsive Website Basics: Code with HTML, CSS, and JavaScript}
\certauthority{University of Michigan}{Coursera}
   \certitembullet{Introduction to HTML5}
   \certitembullet{Advanced Styling with Responsive Design}
\certauthority{The Hong Kong University of Science}{Coursera}
   \certspecialization{Full Stack Web Development Specialization}
      \certitembullet{HTML, CSS and JavaScript}
      \certitembullet{Front-End Web UI Frameworks and Tools}
      \certitembullet{Front-End JavaScript Frameworks: AngularJS}
      \certitembullet{Multiplatform Mobile App Development with Web Technologies}
      \certitembullet{Server-side Development with NodeJS}
      \certitembullet{Full Stack Web Development Specialization Capstone Project}
\certitemoneline{Pluralsight}{Code School}{Shaping up with Angular.js}
\certauthority{Microsoft}{edX}
   \certitembullet{Introduction to JQuery}
   \certitembullet{Introduction to TypeScript}
\certauthority{MongoDB}{MongoDB University}
   \certitembullet{M101JS: MongoDB for Node.js Developers}
%\certauthority{MongoDB}{edX}
   \certitembullet{M101x: Introduction to MongoDB using the MEAN Stack}
\certauthority{at\&t, Google, GitHub, Hack Reactor}{Udacity}
   \certspecialization{Front-End Web Developer Nanodegree}
      \certitembullet{Intro to AJAX}
      \certitembullet{JavaScript Basics}
      \certitembullet{JavaScript Design Patterns}
      \certitembullet{JavaScript Testing}
      \certitembullet{Object-Oriented JavaScript}

\smallskip
\certgroup{OTHER DOMAINS}

\certauthority{University of California, san Diego}{Coursera}
   \certitembullet{How to Learn:~powerful mental tools to help you master tough
             subjects}
\egroup

\vfill
\supereject\end
